% !TEX root = ../Projektdokumentation.tex
\section{Einleitung}
\label{sec:Einleitung}

\subsection{Projektziel}
\label{subsec:Projektziel}

Das Abschlussprojekt hat das Ziel, die BFM-Anwendung zu erweitern, indem die Login-Aktivitäten von Benutzern über die Webapplikation nachverfolgt und übersichtlich im Frontend
dargestellt werden.\ Diese Anpassung soll dem Benutzer Klarheit über seine Kontoaktivitäten verschaffen und gleichzeitig das BFM-Team entlasten, da dadurch erwartet wird, dass
einige Support-Fälle verhindert werden können.\ Im Zuge dessen muss die Anzeige möglichst benutzerfreundlich und einfach zu bedienen sein.

In früheren BFM-Versionen führte die Anzeige eines Zeitstempels mit dem Hinweis \textit{Zuletzt angemeldet} im Dashboard unter Anmeldeinformationen zu Verwirrung bei den Benutzern.\
Mit dem Vermerk, dass sich der Zeitstempel ausschließlich auf die letzte Anmeldung über die Webapplikation bezieht, wurde dieses Problem nur vorläufig behoben.\ Jedoch haben Kunden immer wieder den
Wunsch nach einem Anmeldeverlauf geäußert, auf dem sie folgenden Informationen nachsehen können: Datum, Anmeldeort (IP-Adresse) und Angaben zum verwendeten Client, \zB bei einer Anmeldung über die Webapp
die App oder den SyncClient.

Hauptziele des Projekts sind die Verbesserung der Usability vom BFM durch die Bereitstellung eines Anmeldeverlaufs für die Webapplikation.

\subsection{Teilaufgaben}
\label{subsec:Teilaufgaben}

Das Projekt gliederte sich in mehrere Phasen, darunter Projektplanung, die Analyse des aktuellen Zustands, der Entwurf einer Lösung unter Berücksichtigung %kaufmännischer,%
wirtschaftlicher und technologischer Aspekte, die Umsetzung und die Qualitätsprüfung durch Tests, um die Anforderungen systematisch umzusetzen.\ Die Analyse- und Entwurfsphasen waren besonders wichtig,
da sie die Grundlage für die erfolgreiche Umsetzung des Projekts bildeten.

\subsection{Kundenwünsche}
\label{subsec:Kundenwuensche}

Die Kundenanforderungen umfassten eine benutzerfreundliche und intuitive Oberfläche, die mit dem Corporate-Design des Unternehmens im Einklang steht.\
Zu den wichtigsten Anforderungen gehörten die Speicherung von Datum und Anmeldestandort bei jeder Anmeldung sowie die Bereitstellung der Lösung in einem dedizierten Modul mit einer öffentlichen API,
die von anderen Komponenten der Anwendung bedient werden kann.\ Dies soll eine Überladung bereits vorhandener Module verhindern sowie die Wartbarkeit und Skalierbarkeit der Lösung sicherstellen.
Darüber hinaus wurde als optionale Anforderung das Einholen von Client-Informationen gewünscht.

\subsection{Projektumfeld}
\label{subsec:Projektumfeld}

Als IT-Dienstleister entwickelt doubleSlash seit 25 Jahren innovative Softwarelösungen und bietet eine breite Palette an Dienstleistungen an.\ Das Unternehmen unterstützt seine Kunden während des gesamten
Entwicklungszyklus - von der Explorationsphase über Konzeption und Planung neuer Softwareprojekte bis hin zur Implementierung, Testung und dem laufenden Betrieb.\ Die Firma doubleSlash ist vor allem in der Mobilitätbranche
in verschiedenen Projekten aktiv, erbringt aber auch Leistungen in der Medizintechnik, im Maschinen- und Anlagenbau, im Energiesektor, in der Logistik sowie im öffentlichen Sektor.
Zudem entwickelt und vertreibt der Dienstleister sein eigenes Produkt namens BFM, eine kollaborative Plattform für den sicheren Datenaustausch über Unternehmensgrenzen hinweg, welche die Grundlage für diese Abschlussarbeit
bildet.\ Veranlasst wurde die Produkterweiterung von Kerstin Glökler, wer die Rolle des Product Owners im BFM-Team innehat.\ Das Entwicklungsteam besteht aus mehreren erfahrenen Entwicklern sowie Nachwuchskräften, \ua Auszubildende, duale Studenten
und Praktikanten.\ Das Projektmanagement erfolgt agil, mit regelmäßigen Meetings und Aufgaben, die in zwei- bis vierwöchigen Sprints umgesetzt werden.

\subsection{Projektschnittstellen} 
\label{subsec:Projektschnittstellen}

Die erfolgreiche Umsetzung des Projekts erforderte eine Zusammenarbeit mit verschiedenen Ansprechpartnern.\ Besonders wichtig war der Austausch
mit dem Entwicklungsteam, welches bei technischen Fragen zum Architektur- und Datenmodell sowie Schnittstellen zur Anwendung mit Rat und Tat stand.\ Zu Beginn des Projekts fanden mehrere Meetings mit Frau Glöckler und Herrn Kreutzfeld statt,
um die Projektanforderungen-, ziele sowie die Abgrenzungen zu definieren.\ Der Projektfortschritt wurde in kurzen Sync-Terminen mit dem Projektbetreuer abgestimmt und überprüft.

\subsection{Projektabgrenzung} 
\label{subsec:Projektabgrenzung}

Die Implementierung des Standort-Trackings für den BFM beschränkt sich auf die Weboberfläche.\ Änderungen an externen Komponenten wie mobile Apps oder Backend-Diensten sind nicht Bestandteil dieses Projekts.\ Da es sich um eine Evaluierung und den Entwurf eines Proof of Concepts(PoC) handelt, liegt
der Fokus darauf, erste Erkenntnisse über die Machbarkeit und Optimierungspotenziale zu gewinnen.\ Die Ergebnisse aus diesem Projekt sollen als Grundlage für zukünftige Verbesserungen und die Einführung im Kundensystem dienen.