% !TEX root = Projektdokumentation.tex
\section{Anhang}
\subsection{Detaillierte Zeitplanung}
\label{app:Zeitplanung}

\tabelleAnhang{ZeitplanungKomplett}

\input{Anhang/AnhangLastenheft.tex}
\clearpage

\subsection{Use Case-Diagramm}
\label{app:UseCase}
Use Case-Diagramme und weitere \acs{UML}-Diagramme kann man auch direkt mit \LaTeX{} zeichnen, siehe \zB \url{http://metauml.sourceforge.net/old/usecase-diagram.html}.
\begin{figure}[htb]
\centering
\includegraphicsKeepAspectRatio{UseCase.pdf}{0.7}
\caption{Use Case-Diagramm}
\end{figure}

\input{Anhang/AnhangPflichtenheft.tex}

\subsection{Datenbankmodell}
\label{app:Datenbankmodell}
ER-Modelle kann man auch direkt mit \LaTeX{} zeichnen, siehe \zB \url{http://www.texample.net/tikz/examples/entity-relationship-diagram/}.
\begin{figure}[htb]
\centering
\includegraphicsKeepAspectRatio{database.pdf}{1}
\caption{Datenbankmodell}
\end{figure}
\clearpage

\input{Anhang/AnhangEntwuerfe.tex}
\clearpage
\input{Anhang/AnhangScreenshots.tex}
\input{Anhang/AnhangDoc.tex}
\clearpage
\subsection{Testfall und sein Aufruf auf der Konsole}
\label{app:Test}
\lstinputlisting[language=php, caption={Testfall in PHP}]{Listings/Out/tests.php}
\clearpage
\begin{figure}[htb]
\centering
\includegraphicsKeepAspectRatio{testcase.jpg}{1}
\caption{Aufruf des Testfalls auf der Konsole}
\end{figure}


\subsection{Klasse: ComparedNaturalModuleInformation}
\label{app:CNMI}
Kommentare und simple Getter/Setter werden nicht angezeigt.
\lstinputlisting[language=php, caption={Klasse: ComparedNaturalModuleInformation}]{Listings/Out/cnmi.php}
\clearpage

\subsection{Klassendiagramm}
\label{app:Klassendiagramm}
Klassendiagramme und weitere \acs{UML}-Diagramme kann man auch direkt mit \LaTeX{} zeichnen, siehe \zB \url{http://metauml.sourceforge.net/old/class-diagram.html}.
\begin{figure}[htb]
\centering
\includegraphicsKeepAspectRatio{Klassendiagramm.pdf}{1}
\caption{Klassendiagramm}
\end{figure}
\clearpage

\input{Anhang/AnhangBenutzerDoku.tex}
